\textbf{Shared storage} In our current prototype implemented on PhantomNet~\cite{phantomnet}, all the cloud nodes have a shared physical storage enabled by iSCSI~\cite{iscsi}. This is the default requirement for VM live migration in Xen. During the VM live migration, only the statuses of VMs in the RAM are actually migrated from one node to the other whil the VMs installed in the shared physical storage are kept intact. Since the target node can also access the VM on the physical storage, once the migration is over, the VM can work as if it is still on the original node. Such implementation is quite infeasible in the reality because it means that thousands of hundreds of cloud nodes must share some physical storage, both directly or in some way indirectly. Also, since the VM is not actually moved, when there is I/O tasks, these data has to travel across the network to reach so that the VM can read or write data on the physical storage where it is. These may result in high latency, making our VM migration meaningless. There are some possible solutions we can use to solve problem. The first one is to use Citrix XenServer, which enables storage XenMotion that migrate the VM in the RAM as well as the one on the physical storage. A second one may be using DRBD~\cite{drbd}, a distributed replicate storage system. Which one is more feasible in the real deployment or if there are other solutions are questions we need to answer in the future.

