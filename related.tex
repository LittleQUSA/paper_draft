Cloud computing has long been a hot topic for researchers. Many approaches are proposed for us to better utilize the possible convenience brought by the clouds. Our work draws inspiration from previous work on computing offloading. However, our system is the first one that provides personal computing offloading services in a mobile network.\\\\
\textbf{Cloudlets in mobile network} This paper~\cite{cloudlets} envisions that in the mobile network, whenever people have access to the network, they will have access to a cloud server nearby, in which they can install VMs based on their own needs. Once the VM is created, they can offload some computing tasks to that VM running in the cloud server. No matter this architecture is possible or not, it provides an very interesting use case to us. Assuming that we have some wearable equipment that provides continuous and low-latency services for us, it would be best if we have some private, low-latency server for the equipment which may not be so powerful to do the computing. The idea of this paper is greatly inspired by this use case.\\\\
\textbf{SMOG architecture} SMOG~\cite{smog} is a cloud platform for seamless wide area migration of online games. Unlike web users, some online game players often have strong affinity to particular servers while they required high QoS between clients and these servers. This paper explores the possibility of dynamic migration of cloud computing resources to solve the problem of hosting servers. As a result, SMOG turns out to be working well even in a wide area migration.\\\\
\textbf{SMORE architecture} The issue of alleviating the burden of the core network and enable efficient routing and low-latency services has been identified and stressed by many previous work. SMORE~\cite{smore}, a software-defined networking mobile offloading architecture is one of them that aims at offloading certain traffic from UEs to a nearby cloud server, thus providing really low-latency service. The SMORE architecture deploys data offloading centers at the MTSOs, which fits well in the mobile network. When SMORE is used, a significant RTT improvement cane be seen. Thus assures us the effectiveness of SMORE.